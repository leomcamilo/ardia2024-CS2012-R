
% \pagenumbering{gobble}



\begin{center}
    \includegraphics[angle=0,keepaspectratio,width=3cm]{UFF.png}
    \end{center}
    
    \begin{center}
    \textbf{\fontsize{12}{14}\selectfont 
    UNIVERSIDADE FEDERAL FLUMINENSE\\[0.2cm]
    FACULDADE DE ADMINISTRAÇÃO E CIÊNCIAS CONTÁBEIS\\[0.2cm]
    Programa de Pós-Graduação em Administração\\[0.2cm]
    Mestrado Acadêmico em Administração\\[4.5cm]
    IMPACTO DO BID-ASK SPREAD NA DETERMINAÇÃO DO PREÇO NO MERCADO DE AÇÕES\\[4cm]
    }
    \end{center}
    
    \begin{center}
    \textbf{LEONARDO MELLO CAMILO DA SILVA\\[5.5cm]
    Niterói\\[0.2cm]
    2024
    }
    \end{center}
    \thispagestyle{empty}
    \begin{center}
    
    \textbf{LEONARDO MELLO CAMILO DA SILVA\\[1.5cm]
            IMPACTO DO BID-ASK SPREAD NA DETERMINAÇÃO DO PREÇO NO MERCADO DE AÇÕES\\[5cm]
            }
        
       
        
        \end{center}
    
    
    \begin{quotation}
    \setlength{\leftskip}{7cm}
     \noindent{Dissertação de mestrado apresentada ao Programa de Pós-Graduação em Administração da Faculdade de Administração e Ciências Contábeis da Universidade Federal Fluminense, como requisito  para obtenção do título de Mestre em Administração.\\[1cm]
        Orientador: Prof. Dr. Claudio Henrique da Silveira Barbedo\\[7.5cm]}
    \end{quotation}
    
    \begin{center}
        Niterói/RJ\\[0.2cm]
        2024 
    \end{center}
    
    \newpage
    
    \begin{center}
        \textbf{\Large Resumo}\\[0.2cm]
    \end{center}
    
    
    
    \begin{flushleft}
        \setlength{\parskip}{1cm} % Espaçamento entre parágrafos
        \linespread{1.5}\selectfont % Espaçamento entre linhas
        \hspace*{0cm}\parbox{16.5cm}{
            Este estudo investiga o impacto do spread bid-ask na formação de preços de ativos financeiros no mercado de ações brasileiro. Utilizando uma amostra de 130 ações listadas na B3, aplicamos três metodologias de estimação: Corwin \& Schultz (2012), Abdi \& Ranaldo (2017) e EDGE (Ardia, Guidotti \& Kroencke, 2024), comparando-as com dados da Bloomberg. Os resultados indicam que o método EDGE apresenta desempenho superior, com RMSE e MAE aproximadamente 65\% menores que os demais estimadores. Durante períodos de alta volatilidade, como a pandemia de 2020, todos os estimadores registraram aumento nos spreads, com EDGE e AR demonstrando maior sensibilidade às condições adversas de mercado.
            \\
            \linespread{1.5}\selectfont
            \textbf{Palavras-chave:} \textit{Bid-ask spread; Microestrutura de mercado; Liquidez de mercado; Custos de transação.}
        }
    \end{flushleft}
    \newpage
    
    
    
    % \begin{center}
    %     \textbf{\Large Agradecimentos}\\[0.2cm]
    % \end{center}
    
    % \begin{flushleft}
    %     \setlength{\parskip}{1cm} % Espaçamento entre parágrafos
    %     \linespread{1.5}\selectfont % Espaçamento entre linhas
    %     \hspace*{0cm}\parbox{16.5cm}{
    %         Escrever aqui os agradecimentos.
    %     }
    
    
    % \end{flushleft}
    % \newpage
    
    
    
    
    
    \thispagestyle{empty}
    
        
    