\usepackage{tocloft}
\usepackage{titletoc}
\usepackage{sectsty}
\usepackage{fancyhdr}
\usepackage{etoolbox}
\usepackage{float}
\usepackage[hyphens]{url}
\usepackage[breaklinks]{hyperref}
\usepackage{indentfirst}
\usepackage{amsmath}
\usepackage{amssymb}
\usepackage{mathtools}
\usepackage{caption}

% Define o estilo do cabeçalho
\floatplacement{table}{H}

% Remove recuos e define margens zero
\setlength{\cftbeforetoctitleskip}{0pt}
\setlength{\cftaftertoctitleskip}{1em}

% Define formato dos títulos com espaço para numeração e tamanho 12pt
\titlecontents{section}[2.3em]{\fontsize{12}{14}\selectfont}
{\contentslabel{2.3em}}{}{\titlerule*[1pc]{.}\contentspage}[\vspace{0.5em}]
\sectionfont{\fontsize{12}{14}\selectfont\bfseries}
\subsectionfont{\fontsize{12}{14}\selectfont\bfseries}

\titlecontents{figure}[2.3em]{\fontsize{12}{14}\selectfont}
{\contentslabel{2.3em}}{}{\titlerule*[1pc]{.}\contentspage}[\vspace{0.5em}]

\titlecontents{table}[2.3em]{\fontsize{12}{14}\selectfont}
{\contentslabel{2.3em}}{}{\titlerule*[1pc]{.}\contentspage}[\vspace{0.5em}]

% Formata títulos das listas
\renewcommand{\cfttoctitlefont}{\fontsize{12}{14}\selectfont\textbf}
\renewcommand{\cftloftitlefont}{\fontsize{12}{14}\selectfont\textbf}
\renewcommand{\cftlottitlefont}{\fontsize{12}{14}\selectfont\textbf}

% Lista de equações
\newcommand{\listequationsname}{\fontsize{12}{14}\selectfont\textbf{LISTA DE EQUAÇÕES}}
\newlistof{equations}{equ}{\listequationsname}
\newcommand{\equations}[1]{%
\addcontentsline{equ}{equations}{\protect\numberline{\theequation}#1}\par}

% Configuração da numeração das páginas
\fancypagestyle{plain}{%
    \fancyhf{}%
    \fancyheadoffset[R]{0cm}%
    \renewcommand{\headrulewidth}{0pt}% Remove linha do cabeçalho
    \fancyhead[R]{%
        \ifnum\value{page}<10
            \phantom{\thepage}
        \else
            \thepage
        \fi
    }%
}

\pagestyle{plain}
\setcounter{page}{2}

% Define o recuo da primeira linha dos parágrafos como 1.25 cm
\setlength{\parindent}{1.25cm}

% Espaçamento vertical antes e depois das tabelas
\setlength{\textfloatsep}{1cm}
\setlength{\intextsep}{1cm}

% Define um espaçamento de 1cm entre a tabela e a legenda
\captionsetup[table]{skip=1cm}